\documentclass[onecolumn, draftclsnofoot,10pt, compsoc]{IEEEtran}
\hbadness=1000 % suppress warnings
\usepackage{graphicx}
\usepackage{url}
\usepackage{setspace}
\usepackage{hyperref}
\usepackage{listings}
\usepackage{cite}
\usepackage{geometry}

\usepackage{longtable}

\geometry{textheight=9.5in, textwidth=7in}

% 1. Fill in these details
\def \CapstoneTeamName{		Aerolyzer}
\def \CapstoneTeamNumber{		19}
\def \GroupMemberOne{			Daniel Ross}
\def \GroupMemberTwo{			Logan Wingard}
\def \CapstoneProjectName{		Aerolyzer}
\def \CapstoneSponsorPerson{		Kim Whitehall}


% 2. Uncomment the appropriate line below so that the document type works
\def \DocType{		%Problem Statement
	%Requirements Document
	%Technology Review
	%Design Document
	Progress Report
}

\newcommand{\NameSigPair}[1]{\par
	\makebox[2.75in][r]{#1} \hfil 	\makebox[3.25in]{\makebox[2.25in]{\hrulefill} \hfill		\makebox[.75in]{\hrulefill}}
	\par\vspace{-12pt} \textit{\tiny\noindent
		\makebox[2.75in]{} \hfil		\makebox[3.25in]{\makebox[2.25in][r]{Signature} \hfill	\makebox[.75in][r]{Date}}}}
% 3. If the document is not to be signed, uncomment the RENEWcommand below
\renewcommand{\NameSigPair}[1]{#1}

%%%%%%%%%%%%%%%%%%%%%%%%%%%%%%%%%%%%%%%
\graphicspath{{images/}}
\begin{document}
	\begin{titlepage}
		\pagenumbering{gobble}
		\begin{singlespace}
			\centering
			\includegraphics[height=4cm,natwidth=345,natheight=435]{images/coe_v_spot1.png}
			\hfill 
			% 4. If you have a logo, use this includegraphics command to put it on the coversheet.
			%\includegraphics[height=4cm]{CompanyLogo}   
			\par\vspace{.2in}
			\centering
			\scshape{
				\huge CS Capstone \DocType \par
				{\large\today}\par
				\vspace{.5in}
				\textbf{\Huge\CapstoneProjectName}\par
				\vfill
				{\large Prepared for}\par
				{\Large\NameSigPair{\CapstoneSponsorPerson}\par}
				{\large Prepared by }\par
				Group\CapstoneTeamNumber\par
				% 5. comment out the line below this one if you do not wish to name your team
				\CapstoneTeamName\par 
				\vspace{5pt}
				{\large
					\NameSigPair{\GroupMemberOne}\par
					\NameSigPair{\GroupMemberTwo}\par
				}
				\vspace{20pt}
			}
			\begin{abstract}  
				The Aerolyzer Project aims to deliver a new source of air quality and weather information through leveraging existing weather data and image analysis algorithms.
				When complete, this open-source project shall feature a Python library that uses image classification and third-party weather APIs, displayed with an intuitive web-based user interface.
				This document outlines the software design descriptions for the Aerolyzer Library. 
			\end{abstract}     
		\end{singlespace}
	\end{titlepage}

\tableofcontents
\clearpage

\begin{singlespace}

	\section{Project Purpose}
		Monitoring atmospheric aerosols is important due to their effects on people’s health and the atmosphere's chemical composition and radiation distribution.
		Currently delayed or inaccurate atmospheric reports complicate getting reliable local atmospheric information.
		Unfortunately, aerosols in the atmosphere are constantly changing, and current satellite, aircraft, and ground-based instruments do not simplify data enough for the average person to understand.
		There is currently no way to judge local air quality using regional aerosol data without in-depth atmospheric knowledge.
		Additionally, delayed or inaccurate atmospheric reports complicate getting reliable local atmospheric information.
		The primary objective of the Aerolyzer project is to create a tool that infers local air quality using regional weather data and image analysis.
		The major goals this presents to the project are a quick weather data retrieval, the identification of an image that can be used for color analysis, and the analysis of the colors in an image to estimate the level of aerosols.
	
	\section{Current State}
		
		\subsection{Logan Wingard}
			
		\subsection{Daniel Ross}
			
			
			
	\section{What's left to accomplish}
		
		\subsection{Logan Wingard}
			
		\subsection{Daniel Ross}
			


	\section{Problems}
		
		\subsection{Logan Wingard}
			
		\subsection{Daniel Ross}
			
			
	\section{Interesting Code}
		The following is an excerpt from the horizon detection function.
		It has since been updated with slightly different numbers in order to increase accuracy, but this is what is currently pushed to out master Aerolyzer github.
		It splits the image at the horizon and then splits the sky in half to analyze the histogram data in both the top and bottom half of the sky.
		These are used to further determine if the image is actually valid.
		If the image is valid, it moves onto color analysis with the bottom half of the sky.
		\begin{lstlisting}
color = ('b', 'g', 'r')
b, g, r = cv2.split(img)
dimy, dimx = img.shape[:2]

largest = [0, 0]
it = dimy / 200 #iterations = total number of rows(pixels) / 200
for i in range(dimy / 4, (dimy / 4) * 3, it):   #only looking at the middle half of the image
	ravg = (sum(r[i]) / float(len(r[i])))
	gavg = (sum(g[i]) / float(len(g[i])))
	bavg = (sum(b[i]) / float(len(b[i])))
	avg = (ravg + gavg + bavg) / 3
	pravg = (sum(r[i - it]) / float(len(r[i - it])))
	pgavg = (sum(g[i - it]) / float(len(g[i - it])))
	pbavg = (sum(b[i - it]) / float(len(b[i - it])))
	pavg = (pravg + pgavg + pbavg) / 3
	diff = pavg - avg
	if diff > largest[0]:   #only getting the largest intensity drop.
		largest = [diff,i-(it/2)]
if largest[0] >= 11:
	sky = img[0:largest[1],0:dimx]#cropping out landscape
	h1 = sky[0:(sky.shape[0] / 2),0:dimx]#top half of sky
	h2 = sky[(sky.shape[0] / 2):(sky.shape[0]), 0:dimx]#bottom half

	mask = np.zeros(h1.shape[:2], np.uint8)
	mask[0:(h1.shape[0] / 2), 0:h1.shape[1]] = 255

	for i,col in enumerate(color):
		histr = cv2.calcHist([h1], [i], mask, [255], [0, 255])
		plt.plot(histr, color = col)
		plt.xlim([0,255])

	mask = np.zeros(h2.shape[:2], np.uint8)
	mask[0:(h2.shape[0] / 2), 0:h2.shape[1]] = 255

	for i,col in enumerate(color):
		histr = cv2.calcHist([h2], [i], mask, [255], [0, 255])
		plt.plot(histr, color = col)
		plt.xlim([0, 255])
			\end{lstlisting}	
			Here are the results of this function:\\
			\includegraphics[height=4cm,natwidth=640,natheight=426]{images/horizon_uncropped.jpg}\\
			\includegraphics[height=4cm,natwidth=1281,natheight=537]{images/horizon_cropped.png}\\
			The following is the sun\_position function
			\begin{lstlisting}
def sun_position(exifdict):
	coord = get_coord(exifdict)
	wData = wunderData.get_data(str(coord[0])+","+str(coord[1]))
	sunriseTime = wData['sunrise'].split(':')
	sunsetTime = wData['sunset'].split(':')
	sunriseTarget = (int(sunriseTime[0])*60)+int(sunriseTime[1])
	sunsetTarget = (int(sunsetTime[0])*60)+int(sunsetTime[1])

	hoursTime = (str(exifdict['exif datetimeoriginal']).split(' '))[1].split(':')
	pictureTime = (int(hoursTime[0])*60)+int(hoursTime[1])+int(float(hoursTime[2])/60)

	if ((pictureTime >= (sunriseTarget - 15)) & (pictureTime <= (sunriseTarget + 30))):
		return 1
	elif ((pictureTime >= (sunsetTarget - 15)) & (pictureTime <= (sunsetTarget + 30))):
		return 2
	elif ((pictureTime > (sunsetTarget + 15))|(pictureTime < (sunriseTarget - 15))):
		return 0
	else:
		return 0
			\end{lstlisting}
\end{singlespace}
\clearpage
\bibliographystyle{IEEEtran}
\bibliography{ref}
\end{document}
