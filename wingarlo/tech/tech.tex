\documentclass[onecolumn, draftclsnofoot,10pt, compsoc]{IEEEtran}
\usepackage{graphicx}
\usepackage{url}
\usepackage{setspace}
\usepackage[margin=0.75in]{geometry}
\setlength{\parindent}{0pt}
\usepackage[hidelinks]{hyperref}
\usepackage{listings}
\usepackage{float}

\renewcommand\thesection{\Roman{section}}
\renewcommand\thesubsection{\Alph{subsection}}
\renewcommand\thesubsubsection{\arabic{subsubsection}}

\usepackage{etoolbox}
\patchcmd{\thebibliography}{\refname}{}{}{}


\usepackage{geometry}
\geometry{textheight=9.5in, textwidth=7in}

% 1. Fill in these details
\def \CapstoneTeamName{		Aerolyzer}
\def \CapstoneTeamNumber{		19}
\def \GroupMemberOne{			Logan Wingard}
\def \GroupMemberTwo{			Kin-Ho Lam}
\def \GroupMemberThree{			Daniel Ross}
\def \CapstoneProjectName{		Aerolyzer}
\def \CapstoneSponsorCompany{	NASA JPL}
\def \CapstoneSponsorPerson{		Kim Whitehall, Lewis McGibbney}

% 2. Uncomment the appropriate line below so that the document type works
\def \DocType{		%Problem Statement
				%Requirements Document
				(rough)Technology Review
				%Design Document
				%Progress Report
				}
			
\newcommand{\NameSigPair}[1]{\par
\makebox[2.75in][r]{#1} \hfil 	\makebox[3.25in]{\makebox[2.25in]{\hrulefill} \hfill		\makebox[.75in]{\hrulefill}}
\par\vspace{-12pt} \textit{\tiny\noindent
\makebox[2.75in]{} \hfil		\makebox[3.25in]{\makebox[2.25in][r]{Signature} \hfill	\makebox[.75in][r]{Date}}}}
% 3. If the document is not to be signed, uncomment the RENEWcommand below
\renewcommand{\NameSigPair}[1]{#1}

%%%%%%%%%%%%%%%%%%%%%%%%%%%%%%%%%%%%%%%
\begin{document}
\begin{titlepage}
    \pagenumbering{gobble}
    \begin{singlespace}
    	%\includegraphics[height=4cm]{coe_v_spot1}
        \hfill 
        % 4. If you have a logo, use this includegraphics command to put it on the coversheet.
        %\includegraphics[height=4cm]{CompanyLogo}   
        \par\vspace{.2in}
        \centering
        \scshape{
            \huge CS Capstone \DocType \par
            {\large\today}\par
            \vspace{.5in}
            \textbf{\Huge\CapstoneProjectName}\par
            \vfill
            {\large Prepared for}\par
            \Huge \CapstoneSponsorCompany\par
            \vspace{5pt}
            {\Large\NameSigPair{\CapstoneSponsorPerson}\par}
            {\large Prepared by }\par
            Group\CapstoneTeamNumber\par
            % 5. comment out the line below this one if you do not wish to name your team
            \CapstoneTeamName\par 
            \vspace{5pt}
            {\Large
                \NameSigPair{\GroupMemberOne}\par
                %\NameSigPair{\GroupMemberTwo}\par
                %\NameSigPair{\GroupMemberThree}\par
            }
            \vspace{20pt}
        }
        \begin{abstract}
        % 6. Fill in your abstract    
        	Aerolyzer is a mobile web application capable of processing valid images from users to provide air quality information and aerosol conditions. 
            This document outlines potential tools and software for the Aerolyzer mobile web application.
        \end{abstract}     
    \end{singlespace}
\end{titlepage}
\newpage
\pagenumbering{arabic}
\tableofcontents
% 7. uncomment this (if applicable). Consider adding a page break.
\clearpage

\begin{flushleft}

\section{Introduction}

To-Do


\section{Neural Networks}
\subsection{Available Technologies}
Nerual networks are matrices of functions and floats that weighs each value itself after "learning" from multiple sets of data. It does this using a long, but simple sigmoidal function seen here:\\
$\sigma(x_0*w_0 + x_1*w_1 + ... + x_m*w_m + b)$\\
Where the Ws are the weights, b is the bias, and Xs are the inputs. 
With enough data, a shallow neural network could easily be written and trained in a simple python program using only base python and numpy. The program "learns" by generating random weights and biases, then checks the solution against known data, and adjusts the weights accordingly, until it has the accuracy desired. For deep neural networks (neural networks with two or more hidden layers), one would most likely want to use supervised machine learning such as tensorflow by Google. 

\subsection{Goals}
Nerual networks could use color data extracted from images as input, and through numerous tests, use accurate weights and biases to accurately identify the types of aerosols in the air, and if they could possibly be harmful or not. To train a neural network such as this would take time and a lot of data, though would definitely be doable. 


\subsection{Criteria}
There are two main factors that will determine whether the use of a neural network would be valid or not, and these are the amount of data we have, and the amount of time we have. With much data that has already been classified by hand, the neural network would be as simple to set up as manually inserting data into a long array,and letting the program do its thing. Unfortunatly, classifying the images by hand so that the program could learn can take very long and may not be faster than creating a function that wil analyze the colors. 
\subsection{Selection}
To-Do


\section{Image Scraping}
\subsection{Available Technologies}
Getting enough data to work with is essential to testing the validity of images the Aerolyzer app will receive from its users. Finding these images is more complicated than it would come accross. There are multiplpe python libraries and third party API for image scraping that could supply the data needed. The python library that would work best would most likely be python image scraper 2.0.7.

\subsection{Goals}
What Aerolyzer hopes to get out of the image scraper is a high volume of valid images. Testing the validity of classifiers for percent accuracy requires hundreds, even thousands of images that do not have filters, and have valid exif data. 

\subsection{Criteria}
The image scraper that will ultimately be used has to consistently supply many images that do not have filters and have valid exif data. The most important criteria however is the consistency. An instagram image scraper may supply numerous valid images, though these could be taken when location data was turned off, or have filters applied, making it an impractical image scraper for Aerolyzer.



\subsection{Selection}
To-Do


\section{Python Modules/Libraries}
\subsection{Available Technologies}
PyPi is Python's package index. It is a massive database of packages created by developers to make custom python libraries widely accessable. This is the most valid technology to use for importing python modules.

\subsection{Goals}
Aerolyzer's goal is to create a custom library that will contain all the tools Aerolyzer creates in order to complete the goal of image classification, analysis, etc.

\subsection{Criteria}
The criteria that determines the library setup will be the availability and convenience. If the Aerolyzer library isn't easily accesible, no one will use it, or it would be too much of a hastle to import and use Aerolyzer's horizon detection, color analysis, and others.
\subsection{Selection}
To-Do



\section{Conclusion}
To-Do

\end{flushleft}


\clearpage


\end{document}
