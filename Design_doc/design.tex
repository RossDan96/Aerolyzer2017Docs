\documentclass[onecolumn, draftclsnofoot,10pt, compsoc]{IEEEtran}
\hbadness=1000 % suppress warnings
\usepackage{graphicx}
\usepackage{url}
\usepackage{setspace}
\usepackage{hyperref}
\usepackage{cite}
\usepackage{geometry}
\geometry{textheight=9.5in, textwidth=7in}

% 1. Fill in these details
\def \CapstoneTeamName{		Aerolyzer}
\def \CapstoneTeamNumber{		19}
\def \GroupMemberOne{			Daniel Ross}
\def \GroupMemberTwo{			Kin-Ho Lam}
\def \GroupMemberThree{			Logan Wingard}
\def \CapstoneProjectName{		Aerolyzer}
\def \CapstoneSponsorCompany{	NASA JPL}
\def \CapstoneSponsorPerson{		Kim Whitehall}


% 2. Uncomment the appropriate line below so that the document type works
\def \DocType{		%Problem Statement
	%Requirements Document
	%Technology Review
	Design Document
	%Progress Report
}

\newcommand{\NameSigPair}[1]{\par
	\makebox[2.75in][r]{#1} \hfil 	\makebox[3.25in]{\makebox[2.25in]{\hrulefill} \hfill		\makebox[.75in]{\hrulefill}}
	\par\vspace{-12pt} \textit{\tiny\noindent
		\makebox[2.75in]{} \hfil		\makebox[3.25in]{\makebox[2.25in][r]{Signature} \hfill	\makebox[.75in][r]{Date}}}}
% 3. If the document is not to be signed, uncomment the RENEWcommand below
\renewcommand{\NameSigPair}[1]{#1}

%%%%%%%%%%%%%%%%%%%%%%%%%%%%%%%%%%%%%%%
\begin{document}
	\begin{titlepage}
		\pagenumbering{gobble}
		\begin{singlespace}
			\includegraphics[height=4cm]{coe_v_spot1}
			\hfill 
			% 4. If you have a logo, use this includegraphics command to put it on the coversheet.
			%\includegraphics[height=4cm]{CompanyLogo}   
			\par\vspace{.2in}
			\centering
			\scshape{
				\huge CS Capstone \DocType \par
				{\large\today}\par
				\vspace{.5in}
				\textbf{\Huge\CapstoneProjectName}\par
				\vfill
				{\large Prepared for}\par
				\Huge \CapstoneSponsorCompany\par
				\vspace{5pt}
				{\Large\NameSigPair{\CapstoneSponsorPerson}\par}
				{\large Prepared by }\par
				Group\CapstoneTeamNumber\par
				% 5. comment out the line below this one if you do not wish to name your team
				\CapstoneTeamName\par 
				\vspace{5pt}
				{\large
					\NameSigPair{\GroupMemberOne}\par
					\NameSigPair{\GroupMemberTwo}\par
					\NameSigPair{\GroupMemberThree}\par
				}
				\vspace{20pt}
			}
			\begin{abstract}  
				The Aerolyzer Project aims to deliver a new source of air quality and weather information through leveraging algorithms to characterize aerosol content in the atmosphere.
				When complete, this open-source project shall feature a Python library that uses image classification and third-party weather APIs, displayed with an intuitive web-based user interface.
				This document outlines the software design descriptions for the Aerolyzer Library. 
			\end{abstract}     
		\end{singlespace}
	\end{titlepage}

\section{Table of Contents}
\tableofcontents
\clearpage

\begin{singlespace}
\section{Overview}
	\subsection{Scope}
		Scope Scope Scope
	\subsection{Purpose}
		Purpose Purpose Purpose
	\subsection{Intended Audience}
		Intended Audience Intended Audience Intended Audience
	\subsection{Conformance}
		Conformance Conformance Conformance
\section{Definitions}
	\subsection{Aerosol}\label{def:aerosol}
		Aerosols are tiny particles dispersed throughout the atmosphere.
		These particles originate from natural sources such as volcanic eruptions, and unnatural sources such as pollution. 
		One can visibly see the effects of aerosols in the 'Rayleigh scattering effect' \cite{allen_2015} which visibly reddens sunsets and sunrises.
	\subsection{Acceptable Image}\label{def:accImg}
		An acceptable image is defined as an unedited image of the horizon \ref{def:horizon}.
		Images used in data collection must have valid EXIF \ref{def:exif} meta data, while the images used in training classifiers only require relevant image content.
	\subsection{Horizon}\label{def:horizon}
		The horizon is defined as the line at which the sky and earth's surface appear to meet.
	\subsection{EXIF}\label{def:exif}
		Exchangeable image file format.
	\subsection{Local}\label{def:local}
		Local is defined as the circular area around an observer with the radius being the distance from an observer's position to the horizon.
		For an observer on the ground, this distance is approximately 2.9 miles (4.7 km) and for an observer standing at an elevation of 100 ft (30 m) above ground level this distance is approximately 12.2 miles (19.6 km).
\section{Conceptual model for software design descriptions}
	\subsection{Software design in context}
		Aerolyzer is a web-based application that will provide users with weather and air quality information based on the user's zip code and user submitted images.
		The software being developed is a Python Library that will provide the Aerolyzer application it's core functionality.
	\subsection{Software design descriptions within the life cycle}
		\subsubsection{Influences on SDD preparation}
		\subsubsection{Influences on software life cycle products}
		\subsubsection{Design verification and design role in validation}
\section{Design description information content}
	\subsection{Introduction}
	\subsection{SDD identification}
	\subsection{Design stakeholders and their concerns}
	\subsection{Design views}
	\subsection{Design viewpoints}
	\subsection{Design elements}
		\subsubsection{Design Entities}
		\subsubsection{Design Attributes}
			\subsubsubsection{Name attribute}
			\subsubsubsection{Type attribute}
			\subsubsubsection{Purpose attribute}
			\subsubsubsection{Author attribute}
		\subsubsection{Design Relationships}
		\subsubsection{Design Constraints}
	\subsection{Design overlays}
	\subsection{Design rationale}
	\subsection{Design languages}
\section{Design Viewpoints}
	\subsection{Introduction}
	\subsection{Context viewpoint}
		\subsubsection{Design Concerns}
		\subsubsection{Design Elements}
		\subsubsection{Example languages}
	\subsection{Composition viewpoint}
		\subsubsection{Design Concerns}
		\subsubsection{Design Elements}
			\subsubsubsection{Function attribute}
			\subsubsubsection{Subordinates attribute}
		\subsubsection{Example languages}
	\subsection{Logical viewpoint}
		\subsubsection{Design Concerns}
		\subsubsection{Design Elements}
		\subsubsection{Example languages}
	\subsection{Dependency viewpoint}
		\subsubsection{Design Concerns}
		\subsubsection{Design Elements}
			\subsubsubsection{Dependencies attribute}
		\subsubsection{Example languages}
	\subsection{Information viewpoint}
		\subsubsection{Design Concerns}
		\subsubsection{Design Elements}
			\subsubsubsection{Data attribute}
		\subsubsection{Example languages}
	\subsection{Patterns use viewpoint}
		\subsubsection{Design Concerns}
		\subsubsection{Design Elements}
		\subsubsection{Example languages}
	\subsection{Interface viewpoint}
		\subsubsection{Design Concerns}
		\subsubsection{Design Elements}
			\subsubsubsection{Interface attribute}
		\subsubsection{Example languages}
	\subsection{Structure viewpoint}
		\subsubsection{Design Concerns}
		\subsubsection{Design Elements}
		\subsubsection{Example languages}
	\subsection{Interaction viewpoint}
		\subsubsection{Design Concerns}
		\subsubsection{Design Elements}
		\subsubsection{Example languages}
	\subsection{State Dynamics viewpoint}
		\subsubsection{Design Concerns}
		\subsubsection{Design Elements}
		\subsubsection{Example languages}
	\subsection{Algorithm viewpoint}
		\subsubsection{Design Concerns}
		\subsubsection{Design Elements}
		\subsubsection{Processing Attribute}
		\subsubsection{Example languages}
	\subsection{Resource viewpoint}
		\subsubsection{Design Concerns}
		\subsubsection{Design Elements}
			\subsubsubsection{Resources attribute}
		\subsubsection{Example languages}

\section{Bibliography}
	\bibliographystyle{IEEEtran}
	\bibliography{ref}

\end{singlespace}
\end{document}
