\documentclass[letterpaper,10pt,draftclsnofoot,onecolumn]{IEEEtran}

\usepackage[margin=0.75in]{geometry}

\newcommand\tab[1][1cm]{\hspace*{#1}}

\def\name{Kin-Ho Lam, Logan Wingard, Daniel Ross \group}
\def\team{Aerolyzer }
\def\assign{Problem Statement}
\def\group{(Group 19)}
\def\course{CS 461 Fall 2017}

% Title page
\title{\team \assign}
\author{\name}
\setlength{\parindent}{0pt}
\begin{document}
\maketitle
\vspace*{10em}
\centering
\course
\vspace*{4em}

\begin{abstract} 
Atmospheric aerosols are particles suspended in the atmosphere. These particles originate from natural sources such as volcanic eruptions, and unnatural sources such as fossil fuel combustion. One can visibly see the effects of aerosols in the 'Rayleigh scattering effect' which visibly reddens sunsets and sunrises. The Aerolyzer Project aims to deliver a new source of air quality and weather information through leveraging machine learning algorithms to characterize aerosol content in the atmosphere. When complete, this open-source project shall feature a Python library that uses image classification and third-party weather APIs, displayed with an intuitive web-based user interface.
\end{abstract}

\clearpage
\begin{flushleft}
\section{Problem Definition}
\tab Monitoring atmospheric aerosols is important due to their effects on the atmosphere's chemical composition and radiation distribution. However, the presence of aerosols reduces air quality which can potentially lead to health complications such as bronchitis or respiratory inflammation. Unfortunately, aerosols in the atmosphere are constantly changing, and current satellite, aircraft, and ground-based instruments do not simplify data enough for the average person to understand. There is currently no way to judge local air quality using regional aerosol data without in-depth atmospheric knowledge. Additionally, delayed or inaccurate atmospheric reports complicate getting reliable local atmospheric information. This issue calls for a technology that gathers information about aerosols in a new way, relates current weather information to aerosol data, and presents its information in an intuitive interface.

\section{Proposed Solution}
\tab The primary objective of the Aerolyzer project is to create a tool that infers local air quality using regional weather data and aerosol analysis. To accomplish this goal, the Aerolyzer project requires a Python library that identifies images relevant to aerosols, analyzes acceptable images for aerosol content, stores relevant information for trend analysis, and compiles weather information with its aerosol data. 
\par
\tab Bringing this library's functionality to a mainstream audience calls for an intuitive web front-end UI. This front end UI shall present the Aerolyzer library's weather and air quality inference based on analyzed images or by a requested location. The UI shall also collect user-submitted cell phone images of the horizon, sunset, and sunrise to be fed to the library for analysis. This image analysis shall be performed using computer vision libraries such as OpenCV, or machine learning tools such as Google Tensorflow. The Aerolyzer project shall exclusively serve users in the United States.

\par
\tab Aerosol particles that are close in size or larger than wavelengths of visible light create a scattering effect, at times diffusing into vibrant colors. This allows one to infer an aerosol's size based on an unedited image of a horizon, sunset, or sunrise. The library shall be capable of performing color analysis on acceptable images to estimate aerosol content. Receiving local air quality information generated by the Aerolyzer project is expected to be faster than conventional aerosol sampling methods because analyzing acceptable images is faster and more practical.

\par
\tab The Aerolyzer web application has been partially developed and shall be expanded to present two options to the user: contribute to the Aerolyzer project by uploading their image of the horizon, or look up local air quality by ZIP code. The library shall provide general weather data from a public API such as Wunderground with relation to a user-submitted image or ZIP code. 

\section{Performance Metrics} 
\tab An acceptable image is defined as an unedited image of the horizon, sunset, or sunrise. The horizon is defined as the line at which the sky and earth's surface appear to meet. A sunrise is defined as the colors and light visible in the sky produced by the sun's first appearance in the morning. A sunset is defined as the colors and light visible in the sky produced when the sun disappears in the evening. These definitions are the Aerolyzer library's classification criteria.
\par
\tab When completed, this library shall identify horizons using Aerolyzer's classifier in a test-selection of images with a minimum accuracy of 66\% certainty. Images of horizons that contain sunsets and sunrises shall be identified using Aerolyzer's classifier library with a minimum of 50\% certainty. This certainty shall be verifiable through a series of regression tests to prove detection accuracy. 

\par
\tab To standardize development between different computers and operating systems, a virtual-machine development environment shall be created. This environment shall contain the libraries and tools necessary to run the Aerolyzer Python library as a stand-alone program as well as verify the aforementioned classifier certainty regression tests.

\clearpage

\section*{Signatures}

\subsection*{Kim Whitehall}

\begin{tabular}{ l p{10pt} l }
Signature: && \hspace{0.5cm} \makebox[3in]{\hrulefill} \\ \\[5pt]
Date: && \hspace{0.5cm} \today
\end{tabular}

\subsection*{Kin-Ho Lam}

\begin{tabular}{ l p{10pt} l }
Signature: && \hspace{0.5cm} \makebox[3in]{\hrulefill} \\ \\[3pt]
Date: && \hspace{0.5cm} \today
\end{tabular}

\subsection*{Logan Wingard}

\begin{tabular}{ l p{10pt} l }
Signature: && \hspace{0.5cm} \makebox[3in]{\hrulefill} \\ \\[3pt]
Date: && \hspace{0.5cm} \today
\end{tabular}

\subsection*{Daniel Ross}

\begin{tabular}{ l p{10pt} l }
Signature: && \hspace{0.5cm} \makebox[3in]{\hrulefill} \\ \\[3pt]
Date: && \hspace{0.5cm} \today
\end{tabular}
\end{flushleft}
\end{document}