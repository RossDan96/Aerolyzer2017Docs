\documentclass[letterpaper,10pt,draftclsnofoot,onecolumn]{IEEEtran}

\usepackage[margin=0.75in]{geometry}

\newcommand\tab[1][1cm]{\hspace*{#1}}

\def\name{Kin-Ho Lam, Logan Wingard, Daniel Ross \group}
\def\team{Aerolyzer }
\def\assign{Problem Statement}
\def\group{(Group 19)}
\def\course{CS 461 Fall 2017}

% Title page
\title{\team \assign}
\author{\name}
\setlength{\parindent}{0pt}
\begin{document}
\maketitle
\vspace*{10em}
\centering
\course
\vspace*{4em}

\begin{abstract} 
Atmospheric aerosols are particles suspended in the atmosphere. These particles originate from natural sources such as volcanic eruptions, and unnatural sources such as fossil fuel combustion. One can visibly see the effects of aerosols in the 'Rayleigh scattering effect' of sunlight which visibly reddens sunsets and sunrises. The Aerolyzer Project will present an air quality and weather synopsis so that users can be more aware of the aerosols near them. Aerolyzer consists of a python library, that uses image classification and weather APIs, which will be hosted in a user friendly web application.
\end{abstract}

\clearpage
\begin{flushleft}
\section{Problem Definition}
\tab Monitoring atmospheric aerosols is important; high concentrations directly and indirectly impact the atmosphere's chemical composition and radiation distribution. Ideally the atmosphere would be composed entirely of gases, but the presence of aerosols reduces air quality and extended exposure to aerosols can lead to health complications like bronchitis. Unfortunately, aerosols in the atmosphere are constantly changing, and current satellite, aircraft, and ground-based instruments do not compile data into language easily understandable by the average person. Currently, there is no way to judge local air quality using regional aerosol data without in-depth atmospheric knowledge. Additionally, delayed or inaccurate atmospheric reports complicate getting reliable local atmospheric information. This issue calls for a tool that gathers information about aerosols in a new way while relating current weather information in an intuitive interface.

\section{Proposed Solution}
\tab The Aerolyzer project aims to estimate local air quality using regional weather data and aerosol analysis for users in the United States. This can be accomplished by crowd-sourcing cell phone images of the horizon, sunsets, and sunrises into a central database for analysis. The primary objective of this project is to create a python library that extracts location data from images and analyzes said images for aerosol content using the OpenCV python library (an open source computer vision library). The secondary goal of the Aerolyzer project is to develop an application that collects images from users and delivers the python library in an intuitive web-based UI. 
\par
\tab The library shall be able to provide general weather data amassed from public API regardless of whether the user provides an image or their zip code. The library  shall be capable of performing color analysis on images that have been classified as sunrises or sunsets to estimate aerosol content. Based on this color analysis, the application shall be able to display the aerosol that most likely caused the scattering of different hues.
\par
\tab The Aerolyzer web application has been partially developed and will be expanded to present two options to the user: contribute to the Aerolyzer project by uploading their image of the horizon, or look up local air quality by zip code. The Aerolyzer project will only serve users in the United States. The Aerolyzer server-sided python library shall use an image classification library to filter out images that do not contain the horizon during sunrises or sunsets. Acceptable images shall be stored and categorized in a database for analysis. 
\par
\tab When a user submits an image, the library will first determine if the image is classified as an acceptable image and then check if the image has valid EXIF metadata. If accepted, the image's zip code is identified via its longitude and latitude. Images will be categorized by zip code and then time for any future referencing. When a user requests information for a zip code, they will receive all the information relevant to that zip code that the library already had access to. In either user story, the python library will call on 3rd party weather APIs, including Wunderground, for their data on the relevant Zip Code.
\par
\tab The image's color profile will be analysed using using OpenCV to compare the hues present in the image to a predetermined data set which correlates certain colors to the presence of aerosols. Aerosols scatter light with a wavelength that is larger than its particle diameter, so the size of the particles can be inferred from the colors in the image. This will add to the relevant data set for that image's zip code. 

\section{Performance Metrics} 
\tab The horizon is defined as the line at which the sky and earth's surface appear to meet. A sunrise is defined as the colors and light visible in the sky produced by the sun's first appearance in the morning. A sunset is defined as the colors and light visible in the sky produced when the sun disappears in the evening. These definitions are the criteria that the library’s classification functions will based on. The classification functions are going to be utilizing the OpenCV library for it’s image interpretation and machine learning functionality.
\pars
\tab When completed, this library shall identify horizons using Aerolyzer's classifier library in a test-selection of images with a minimum accuracy of 50\% certainty. Images of horizons that contain sunsets and sunrises shall be identified using Aerolyzer's classifier library with a minimum of 50\% certainty. This certainty shall be verifiable through a series of regression tests to prove the accuracy of horizon detection. 
\par
\tab To standardize development between different computers and operating systems, a virtual-machine development environment shall be created. This environment shall contain the libraries and tools necessary to run the Aerolyzer Python library as a stand-alone program as well as  verify the aforementioned classifier certainty regression tests.


\clearpage

\section*{Signatures}

\subsection*{Kim Whitehall}

\begin{tabular}{ l p{10pt} l }
Signature: && \hspace{0.5cm} \makebox[3in]{\hrulefill} \\ \\[5pt]
Date: && \hspace{0.5cm} \today
\end{tabular}

\subsection*{Kin-Ho Lam}

\begin{tabular}{ l p{10pt} l }
Signature: && \hspace{0.5cm} \makebox[3in]{\hrulefill} \\ \\[3pt]
Date: && \hspace{0.5cm} \today
\end{tabular}

\subsection*{Logan Wingard}

\begin{tabular}{ l p{10pt} l }
Signature: && \hspace{0.5cm} \makebox[3in]{\hrulefill} \\ \\[3pt]
Date: && \hspace{0.5cm} \today
\end{tabular}

\subsection*{Daniel Ross}

\begin{tabular}{ l p{10pt} l }
Signature: && \hspace{0.5cm} \makebox[3in]{\hrulefill} \\ \\[3pt]
Date: && \hspace{0.5cm} \today
\end{tabular}
\end{flushleft}
\end{document}