\documentclass[letterpaper,10pt,draftclsnofoot,onecolumn]{IEEEtran}

\usepackage[margin=0.75in]{geometry}

\def\name{Kin-Ho Lam, Logan Wingard, Daniel Ross \group}
\def\team{Aerolyzer }
\def\assign{Problem Statement}
\def\group{(Group 19)}
\def\course{CS 461 Fall 2017}

% Title page
\title{\team \assign}
\author{\name}
\setlength{\parindent}{0pt}
\begin{document}
\maketitle
\vspace*{10em}
\centering
\course
\vspace*{4em}

\begin{abstract} 
Atmospheric aerosols are tiny particles suspended in the atmosphere. These particles originate from natural sources such as volcanic eruptions and unnatural sources such as sulfate aerosols from the combustion of fossil fuels. High concentrations of aerosols can impact the atmosphere's chemical composition and radiation distribution. One can visibly see the effects of aerosols in their Rayleigh scattering of sunlight which visibly reddens sunsets and sunrises. Data on aerosols is obtained through a combination of satellite, aircraft, and ground-based instruments. In the context of a typical citizen, the data collected by these mediums is largely unavailable or ambiguous to understand. Currently, there is no way to judge local air quality using regional aerosol data without in-depth atmospheric knowledge. Additionally, delayed or inaccurate atmospheric reports complicate getting reliable local atmospheric information. Our proposal to bridge this information-gap involves crowd-sourcing local cell phone images of sunsets and sunrises to analyze the horizon's colors.
\end{abstract}

\clearpage
\begin{flushleft}
\section{Problem Definition}
The Aerolyzer project is primarily a python library that provides local air quality information based on zip codes by analyzing images of the horizon. The secondary goal of the Aerolyzer project is to develop a web-based application that delivers the python library in an easy to use UI. The atmospheric analysis shall be presented to the user in a clear set of visualizations and values. So in summation this project will focus on images where aerosol analysis can be performed, utilizes existing weather APIs to extrapolate greater information about the local weather, and compiles these two sets of data into a central database.

\section{Proposed Solution}
The web-end UI shall present two options to the user: contribute to the Aerolyzer project by uploading their image of the horizon, or look up local air quality by zip code. Designating data by it's associated zip code means that the application will only be developed for the US. The Aerolyzer python library shall use a classification algorithm to filter out images that do not contain the horizon during sunrises or sunsets. Acceptable images shall be stored and categorized in a database for future analysis. Users who want to view an estimate of their local air quality shall be able to input their zip code into the Aerolyzer web application. The server-side python API shall then perform an analysis on relevant images in its database based on date, location, and 3rd party APIs. \par The Aerolyzer python library shall be able to identify sunrises, sunsets, the horizon, and atmospheric phenomena. The library hopes to gather additional information on aerosols by analyzing the color of the sky. The Aerolyzer web application, that will utilize the python library, has already been developed in Django and will be expanded in order to properly display the atmospheric data in a usable UI. Images and other data from available sources will assist in the inferences of atmospheric aerosol composition, as well as aid in determining the type of atmospheric phenomenon the image displays.

\section{Performance Metrics} (add: accuracy of horizon detection 50\%) (aerosol composition guessing)
When completed, this library shall identify horizons in a test-selection of images with a minimum accuracy of 66\%. Of pictures that have had horizons identified, sunsets and sunrises will be identified with a minimum of 50\% certainty. This certainty shall also be verifiable through a series of regression tests to prove the accuracy of horizon detection. The library will be able to provide general weather data amassed from public API regardless of whether the user provides a image or their zip code. The color analysis of user input images will only be available on images that are identified as sunsets or sunrises. Based on color analysis the application will return the Aerosol that most likely caused the scattering of different hues. To standardize future development, a development environment shall be created.

\clearpage

\section*{Signatures}

\subsection*{Kim Whitehall}

\begin{tabular}{ l p{10pt} l }
Signature: && \hspace{0.5cm} \makebox[3in]{\hrulefill} \\ \\[5pt]
Date: && \hspace{0.5cm} \today
\end{tabular}

\subsection*{Kin-Ho Lam}

\begin{tabular}{ l p{10pt} l }
Signature: && \hspace{0.5cm} \makebox[3in]{\hrulefill} \\ \\[3pt]
Date: && \hspace{0.5cm} \today
\end{tabular}

\subsection*{Logan Wingard}

\begin{tabular}{ l p{10pt} l }
Signature: && \hspace{0.5cm} \makebox[3in]{\hrulefill} \\ \\[3pt]
Date: && \hspace{0.5cm} \today
\end{tabular}

\subsection*{Daniel Ross}

\begin{tabular}{ l p{10pt} l }
Signature: && \hspace{0.5cm} \makebox[3in]{\hrulefill} \\ \\[3pt]
Date: && \hspace{0.5cm} \today
\end{tabular}
\end{flushleft}
\end{document}
