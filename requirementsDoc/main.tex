\documentclass[journal,10pt,draftclsnofoot,onecolumn]{IEEEtran}
\usepackage[doublespacing]{setspace}
\usepackage[margin=0.75in]{geometry}

\begin{document}

\begin{titlepage}
\newcommand{\HRule}{\rule{\linewidth}{0.4mm}}

\center

\textsc{\Large CS461 Fall 2017 }\\[0.5cm]
\textsc{\large Group 19}\\[0.5cm] % Minor heading such as course title

{\huge\bfseries Software Requirement Specifications for Aerolyzer Python API}\\[0.4cm]

\vfill
\begin{minipage}{0.4\textwidth}
\begin{flushleft}
\large
\textit{Authors}\\
Kin-Ho \textsc{Lam}\\
Logan \textsc{Wingrad}\\
Daniel \textsc{Ross}\\
\end{flushleft}
\end{minipage}

\begin{minipage}{0.4\textwidth}
\begin{flushright}
\large
\textit{Supervisor}\\
Dr. Kim \textsc{Whitehall}
\end{flushright}
\end{minipage}

\vfill

\begin{abstract}
The following document details the software requirement specifications for Aerolyzer. Aerolyzer is a web application which hopes to use cell phone images to infer local atmospheric phenomena in the United States. 
\end{abstract}

\vfill\vfill\vfill
{\large10/25/17}
\vfill
\end{titlepage}

\tableofcontents
\clearpage

\begin{singlespace}

\section{(Introduction)}

\subsection{Purpose}
The primary objective of the Aerolyzer project is to create a tool that infers local air quality using regional weather data and aerosol analysis. To accomplish this goal, the Aerolyzer project requires a Python library that identifies images relevant to aerosols, analyzes acceptable images for aerosol content, stores relevant information for trend analysis, and compiles weather information with its aerosol data.

\subsection{Scope}
A front-end UI has already been developed, but must be modified in order to become a useful too. Bringing this library's functionality to a mainstream audience calls for an open-source back-end Python API to perform image analysis. This front end UI shall present the Aerolyzer library's weather and air quality inference based on analyzed images or by a requested location. The UI shall also collect user-submitted cell phone images of the horizon, sunset, and sunrise to be fed to the library for analysis. The Aerolyzer project shall exclusively serve users in the United States.

\subsection{Definitions}
Aerosol: Small particles suspended in the atmosphere. One can visibly see the effects of aerosols in the 'Rayleigh scattering effect' which visibly reddens sunsets and sunrises. \\

Acceptable image: An acceptable image is defined as an unedited image of the horizon, sunset, or sunrise.\\

Horizon: The horizon is defined as the line at which the sky and earth's surface appear to meet.\\

Sunrise: A sunrise is defined as the colors and light visible in the sky produced by the sun's first appearance in the morning.\\

Sunset: A sunset is defined as the colors and light visible in the sky produced when the sun disappears in the evening.\\

Exif data: Exchangeable image file format.\\

\subsection{Overview}

\subsection{References}
\section{(Requirement Description)}
The following will give an in depth description of what the Aerolyzer app will do. It will discuss any constraints the application may have, assumptions we have on our users, and the product functionality.
\subsection{Perspective}

\subsection{Functionality}
Aerolizer will be a smartphone application that grants its users information on aerosols in the atmosphere based on collected information from images taken by users as well as the exif data that came along with the image. It will use machine learning to analyze the wavelengths of the light refracting through our atmosphere to determine the size of the aerosols in the air. It will then relay the information to the user in a user-friendly UI. 
\subsection{Expected user characteristics}
\begin{enumerate}
\item The user will own a smartphone
\item The user has a general idea of what aerosols are
\item 
\end{enumerate}

\subsection{Assumptions}
\begin{enumerate}
\item We assume the user has access to the internet
\item We assumer the user had location enabled when an image was taken
\item We assume the images are not filtered
\end{enumerate}

\subsection{Constraints}
\begin{enumerate}
\item The image must be taken on a device with location tracking enabled
\item The image must be unedited and have no filters applied
\item The image must also have exif data
\item The image must contain a horizon
\item The image must contain either a sunset or sunrise
\item The image must be from in the USA

\end{enumerate}
\nocite{*}
\bibliographystyle{plain}
\bibliography{ref}


\end{singlespace}
\end{document}
